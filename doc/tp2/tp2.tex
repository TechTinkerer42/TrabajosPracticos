% Preview source code

%% LyX 1.6.4.1 created this file.  For more info, see http://www.lyx.org/.
%% Do not edit unless you really know what you are doing.
\documentclass[english]{article}
\usepackage[T1]{fontenc}
\usepackage[latin9]{inputenc}
\usepackage{longtable}

\makeatletter

%%%%%%%%%%%%%%%%%%%%%%%%%%%%%% LyX specific LaTeX commands.
%% Because html converters don't know tabularnewline
\providecommand{\tabularnewline}{\\}

\makeatother

\usepackage{babel}

\begin{document}
\begin{enumerate}
\item Matriz de probabilidades de gustos para cada tipo de oyente de la
radio:


\begin{tabular}{|c|c|c|}
\hline 
 & J & V\tabularnewline
\hline
\hline 
P(1) & 0.95 & 0.03\tabularnewline
\hline 
P(2) & 0.05 & 0.82\tabularnewline
\hline 
P(3) & 0.02 & 0.34\tabularnewline
\hline 
P(4) & 0.2 & 0.92\tabularnewline
\hline
\end{tabular}

Condicion $c$ = El oyente disfruta del los programas 1 y 3, pero
no de los programas 2 y 4.

Sabiendo que un cumple con $c$. Se pueden definir las siguientes
variables:

$x_{j}=$ Oyentes jovenes que cumplen con $c$.

$x_{v}=$ Oyentes viejos que cumplen con $c$.

Con estas ultimas formulas recien definidias, resulta facil ver que
cualquier oyente que cumple con $c$, tiene que pertenecer al conjunto
$x_{j}$ o $x_{v}$. Por lo que se podria definir ademas la varaible$x_{Total}=x_{j}+x_{v}$
(notar que $x_{j}\cap x_{v}=\phi$)

Quedando asi la siguiente formula:

\begin{equation}
P(J|c)=\frac{x_{j}}{x_{Total}}=\frac{x_{j}}{x_{j}+x_{v}}\end{equation}


En donde cada $x$ se expresa como:

$x_{j}=P(1|J)*P(2|J)*(1-P(3|J))*(1-P(1|J))$

$x_{v}=P(1|V)*P(2|V)*(1-P(3|V))*(1-P(1|V))$

Reemplazando las formulas con los valores entregados:

$x_{j}=0.95*0.02*(1-0.02)*(1-0.2)\approx0.015$

$x_{v}=0.03*0.82*(1-0.34)*(1-0.92)\approx0.001$

Finalmente, si reemplazamos estos ultimos resultados en $(1)$, resulta
que \[
P(J|c)=\frac{0.015}{0.015+0.001}=0.92\]


Es decir, hay un $92\%$ de probabilidad que el oyente sea joven.

\item Algormitmo $h_{MAP}$


Para cada hip�tesis h de H, se calcula: $P(h|D)=\frac{P(D|h)*P(h)}{P(D)}$

Se da como salida $h_{MAP}$ = ${max\atop h\epsilon H}P(h|D)$ \\


Hip�tesis a consderar:

$x=(1,0,1,1,0)$

$h_{1}=$La persona es escoses

$h_{2}=$La persona es inglesa.

Calculo de probabilidad para cada hip�tesis:\\


Se define la probabilidad de $h$ como: $P(h)=\frac{1}{|H|}=0.5=h_{1}=h_{2}$
\begin{itemize}
\item $P(h_{1}|D)=\frac{1*0.5}{P(D)}=0.5$
\item $P(h_{2}|D)=\frac{1*0.5}{P(D)}=0.5$
\end{itemize}
Con los datos observados el algoritmo $h_{max}$ no es capaz de asegurar
si $x$ es escoses o ingles ya que ambos tienen la misma probabilidad
de ocurrir.

\item Solucion


\begin{longtable}{|c||c|c|c|c|c|}
\hline 
 & Maiz & Granola & Azucarados & Avena & Mayor a 60\tabularnewline
\hline
\hline 
1 & 1 & 0 & 0 & 0 & 1\tabularnewline
\hline 
2 & 1 & 0 & 0 & 1 & 1\tabularnewline
\hline 
3 & 1 & 1 & 1 & 1 & 1\tabularnewline
\hline 
4 & 0 & 0 & 0 & 1 & 1\tabularnewline
\hline 
5 & 0 & 1 & 1 & 0 & 0\tabularnewline
\hline 
6 & 1 & 1 & 0 & 0 & 0\tabularnewline
\hline
\end{longtable}

Se desea clasificar la instancia: $x=(0,1,1,0)$.

Formula del clasificador de Naive de Bayes\[
v_{NB}={max\atop v_{j}\varepsilon V}P(v_{j})\prod_{i=0}^{n}P(a_{i}|v_{j})\]

\begin{itemize}
\item Calculo de la probabilidad


$v_{NB}=$P($v_{j}$)P(Maiz = 0|$v_{j}$)P(Granola = 1|$v_{j}$)P(Azucarado
= 1|$v_{j}$)P(Avena = 0|$v_{j}$)

P(Mayor a 60 = 1) = $\frac{4}{6}=0.667$

P(Mayor a 60 = 0) = 1 - P(Mayor a 60 = 1) = $\frac{2}{6}=0.333$

P(Maiz = 0|Mayor a 60 = 1) = $\frac{1}{4}=0.25$

P(Maiz = 0|Mayor a 60 = 0) = $\frac{1}{2}=0.5$

P(Granola = 1|Mayor a 60 = 1) = $\frac{1}{4}=0.25$

P(Granola = 1|Mayor a 60 = 0) = $\frac{2}{2}=1$

P(Azucarado = 1|Mayor a 60 = 1) = $\frac{1}{4}=0.25$

P(Azucarado = 1|Mayor a 60 = 0) = $\frac{1}{2}=0.5$

P(Avena = 0|Mayor a 60 = 1) = $\frac{1}{4}=0.25$

P(Avena = 0|Mayor a 60 = 0) = $\frac{2}{2}=1$\\


Quedando las ecuaciones de las probabilidades de $x$ de la siguiente
manera:

Sea $c$ = Mayor a 60.

Sea $d$ = No es mayor a 60.

(1) - P($c$)P(Maiz = 0|$c$)P(Granola = 1|$c$)P(Azucarado = 1|$c$)P(Avena
= 0|$c$) = $0.667*0.25*0.25*0.25*0.25=2.6x10^{-3}$

(2) - P($d$)P(Maiz = 0|$d$)P(Granola = 1|$d$)P(Azucarado = 1|$d$)P(Avena
= 0|$d$) = $0.333*0.5*1*0.5*1=0.083$\\


Por ser el resultado de $(1)>(2)$, el algoritmo escoge $v_{NB}=x$
es Mayor a 60 con una probabilidad de $\frac{(1)}{(1)+(2)}\simeq97\%$.
Cabe aclarar que este resultado fue obtenido a partir de una muestra
muy $chica$, por lo que su grado de certeza podria no ser ser aceptable.

\end{itemize}
\item Solucion


\begin{tabular}{|c||c|c|c|c|}
\hline 
 & Rico & Casado & Saludable & Contenta?\tabularnewline
\hline
\hline 
1 & 1 & 1 & 1 & 1\tabularnewline
\hline 
2 & 0 & 0 & 1 & 1\tabularnewline
\hline 
3 & 1 & 1 & 0 & 1\tabularnewline
\hline 
4 & 1 & 0 & 1 & 1\tabularnewline
\hline 
5 & 0 & 0 & 0 & 0\tabularnewline
\hline 
6 & 1 & 0 & 0 & 0\tabularnewline
\hline 
7 & 0 & 0 & 1 & 0\tabularnewline
\hline 
8 & 0 & 1 & 0 & 0\tabularnewline
\hline 
9 & 0 & 0 & 0 & 0\tabularnewline
\hline
\end{tabular}\\
Se desea calcular la probabilidad que la instancia: $x=(0,1,1)$
este feliz con su vida.
\begin{itemize}
\item Calculo de la probabilidad


$v_{NB}=$P($v_{j}$)P(Rico = 0|$v_{j}$)P(Casado = 1|$v_{j}$)P(Saludable
= 1|$v_{j}$)

P(Contenta = 1) = $\frac{4}{9}=0.444$

P(Contenta = 0) = $\frac{5}{9}=0.556$

P(Rico = 0|Contenta = 1) = $\frac{1}{4}=0.25$

P(Rico = 0|Contenta = 0) = $\frac{4}{5}=0.8$

P(Casado = 1|Contenta = 1) = $\frac{2}{4}=0.5$

P(Casado = 1|Contenta = 0) = $\frac{1}{5}=0.2$

P(Saludable = 1|Contenta = 1) = $\frac{3}{4}=0.75$

P(Saludable = 1|Contenta = 0) = $\frac{1}{5}=0.2$

Ecuaciones de las probabilidades de $x$ de la siguiente manera:

Sea $c$ = Esta contenta con su vida.

Sea $d$ = No esta contenta con su vida.

1$\Longrightarrow$ P($c$)P(Rico = 0|$c$)P(Casado = 1|$c$)P(Saludable
= 1|$c$) = $0.444*0.25*0.5*0.75=0.042$

2$\Longrightarrow$ P($d$)P(Rico = 0|$d$)P(Casado = 1|$d$)P(Saludable
= 1|$d$) = $0.556*0.8*0.2*0.2=0.0178$
\begin{enumerate}
\item El alorimto retorna que la persona esta contenta con una probabilidad
de acierto de $\frac{0.042}{0.042+0.0178}=0.70$. Es decir, un $70\%$.
\item Sea una persona /$x=(0,1,?)$. La probabilidad de que $x$ este Contenta,
esta dada por:\\
-$x_{1}=(0,1,0)$ este contenta o que $x_{2}=(0,1,1)$ este contenta.


$c$ = la persona esta contenta

P(Saludable = 0|$c$) = $\frac{1}{4}=0.25$

P($c$)P(Rico = 0|$c$)P(Casado = 1|$c$)P(Saludable = 0|$c$) = $0.444*0.25*0.5*0.25=0.014$

P($c$)P(Rico = 0|$c$)P(Casado = 1|$c$)P(Saludable = 1|$c$) = $0.444*0.25*0.5*0.75=0.042$

Por lo tanto, la probabilidad que una persona Pobre y Casada este
contenta esta dada por $p=0.014+0.042=0.434$

\item adsadasds
\end{enumerate}
\end{itemize}
\item asd
\end{enumerate}
df
\end{document}

