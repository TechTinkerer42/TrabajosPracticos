% Vista preliminar del código fuente

%% LyX 2.0.0 created this file.  For more info, see http://www.lyx.org/.
%% Do not edit unless you really know what you are doing.
\documentclass[english]{article}
\usepackage[T1]{fontenc}
\usepackage[latin9]{inputenc}
\usepackage[a4paper]{geometry}
\geometry{verbose,tmargin=2cm,bmargin=3cm,lmargin=2cm,rmargin=2cm}
\usepackage{float}
\usepackage{textcomp}
\usepackage{amstext}
\usepackage{graphicx}

\makeatletter

%%%%%%%%%%%%%%%%%%%%%%%%%%%%%% LyX specific LaTeX commands.
%% Binom macro for standard LaTeX users
\newcommand{\binom}[2]{{#1 \choose #2}}

%% Because html converters don't know tabularnewline
\providecommand{\tabularnewline}{\\}

%%%%%%%%%%%%%%%%%%%%%%%%%%%%%% Textclass specific LaTeX commands.
\newenvironment{lyxcode}
{\par\begin{list}{}{
\setlength{\rightmargin}{\leftmargin}
\setlength{\listparindent}{0pt}% needed for AMS classes
\raggedright
\setlength{\itemsep}{0pt}
\setlength{\parsep}{0pt}
\normalfont\ttfamily}%
 \item[]}
{\end{list}}

\makeatother

\usepackage{babel}
\begin{document}

\title{Aprendizaje Automático - Trabajo Práctico 3}


\author{Gonzalo Castiglione - 49138}

\maketitle

\paragraph*{Objetivo: Aplicar diversos métodos estadísticos para aprender a hacer
inferencia a partir de datos experiemtales.}


\section{Métodos de estadística paramétrica}
\begin{enumerate}
\item Soluciones

\begin{enumerate}
\item medidas:
\item %
\begin{tabular}{|c|c|c|c|c|}
\hline 
 & Largo Cépalo & Ancho Cépalo & Largo Pétalo & Ancho Pétalo\tabularnewline
\hline 
\hline 
$\hat{\mu}$ &  &  &  & \tabularnewline
\hline 
$\hat{\sigma}$ &  &  &  & \tabularnewline
\hline 
$E_{cm}$ &  &  &  & \tabularnewline
\hline 
\end{tabular}
\end{enumerate}

5.0060 3.4280 1.4620 0.2460 


5.9360 2.7700 4.2600 1.3260 


6.5880 2.9740 5.5520 2.0260
\begin{enumerate}
\item Intervalos de confianza para con un nivel de confianza de $0.95$.


\begin{tabular}{|c|c|c|c|c|}
\hline 
 & Ancho & Largo & Ancho & Largo\tabularnewline
\hline 
\hline 
$I$ & $2.8643$ & $1.9657$ & $3.0837$ & $2.0863$\tabularnewline
\hline 
\end{tabular}

\end{enumerate}
\item Se tienen 80 componentes, de las cuales 12 son defectuosas. Por ser
este experimiento una secuencia de ensayos Bernoulli, repetidos n
veces, se lo puede considerar una distribución binomial.

\begin{enumerate}
\item La proporción de componentes no defetuosos de la muestra = $\bar{x}_{nd}\frac{80-12}{80}=0.85$

\begin{enumerate}
\item Un estimador $\hat{x}$ es un estimador insesgado para estimar a $x$
si $E[\hat{x}]=p$.


Sea $S_{n}=x_{1}+x_{2}+...+x_{n}$. En donde cada $x_{i}$ representa$1$
si el componente no esta defectuoso o $0$ en caso contrario.


$E[\bar{x}]=E((\sum x_{i})/n)=\frac{1}{n}\sum E(x_{i})=\frac{1}{n}np=p$. 


Por lo tanto este es un estimador $insesgado$.

\item Muestra: $68$ mediciónes con $\{x_{i},y_{i}\}=1$ y $12$ mediciónes
con $\{x_{i},y_{i}\}=0$. 


$e_{0}=(0-0.85)$ (para las $12$ muestras defectuosas)


$e_{1}=(1-0.85)$ (para las $68$ muestras no defectuosas)


Por lo que el error cuádratico medio estaría dado por la fórmula:


$E_{CM}=\sqrt{\frac{(1-0.85)\text{\texttwosuperior}*68+(0-0.85)\text{\texttwosuperior}*12}{80}}=\sqrt{\frac{1.53+8.67}{80}}\simeq0.35$

\end{enumerate}
\item Proporción de sistemas que funcionan correctamente = $\frac{\binom{80-12}{2}}{\binom{80}{2}}=\frac{2278}{3160}=0.72$
\end{enumerate}
\item código:

\begin{lyxcode}
birds={[}~

...replace~meditions~here...

{]};

alpha=0.01~

{[}h,p,ci,stats{]}~=~ttest(birds(:,2),~birds(:,3),~alpha)~\\

\end{lyxcode}

Si alpha > p-value = 0.0117 


~~~~Entonces se rechaza $H_{0}$ (o sea que uD=0) 


Si no


~~~~Entonces no se rechaza, es decir las plumas podrían ser iguales.


Estan en lo cierto, para alpha $=0.05$ hay variación entre el color
de las plumas

\item Solución

\begin{enumerate}
\item .

\begin{enumerate}
\item Código de matlab utilizado

\begin{lyxcode}
load~brains.txt;

x~=~1:28;

y~=~brains(:,1);

z~=~brains(:,2);

clf;

hold~on;

plot(x,~y,~'{*}b;Peso~Promedio~en~Kg;')

plot(x,~z,~'{*}r;Peso~Cerebro~Promedio~en~G;')

print('-dpng',~'./TotalWeightVsBrainWeight.png')~\\

\end{lyxcode}

(Ver figura 1 y 2 - Anexo)


En una observación a simple vista, se puede ver que las mediciónes
que se diferencian notablemente del resto son: $6,7,14,15,16$ y por
supuesto, la $25$.

\item No
\end{enumerate}
\item Código de matlab utilizado

\begin{lyxcode}
load~brains.txt~

regstats(brains(:,1),~brains(:,2),'línear')~\\

\end{lyxcode}

(Ver figura 3 - Anexo)

\item La recta que mejor ajusta con los ejes $x$ e $y$ con la función
$log$ aplicados a ambos es: $y=a*x+b$. Con $a=0.496$ y $b=2.55$.
Por lo tanto, para obtener la recta que ajusta a los puntos $x$ e
$y$ sin aplicar las transformación, es aplicar la tranformación inversa
al $log$, es decir $pow(10,n)$. Quedando así la curva que aproxima
a los puntos como: $10^{y}=a*10^{x}+b$. Despejando por $y$, $y=log(a*10^{x}+b)$.
En la figura 4 del Anexo se puede ver cuan bien se aproximan los valores.
\item En los gráficos del punto $b$, se puede observar que las mediciones
que estan muy fuera del común de son no solo las$14,15$ y $25$ sinó
que también la $6$ y la $16$. Por lo que fueron removidos de la
tabla de valores. El gráfico obtenido luego de ajustados los valores
se muestra en la fiegura $5$ del Anexo.
\end{enumerate}
\item asd
\item Estimadores de maxima verosimilitud


mu hat:


5.8433 3.0573 3.7580 1.1993 \\



sigma hat: 


0.4214 0.0756 2.9140 0.5361

\end{enumerate}
\begin{lyxcode}
\newpage{}
\end{lyxcode}

\section{Anexo}

Gráficos de cada ejercicio:

1)

\begin{figure}[H]
\hspace{1.5cm}\includegraphics[scale=0.2]{TotalWeightVsBrainWeight}\caption{Peso del cerebro y peso total para cada medición en brains.txt{*}}
\end{figure}


{*} El valor del peso del cerebro de la medición 25 no se ve en la
figura ya que se aleja demasiado del resto de los valores y el ajustar
los ejes solo para mostrar ese valor produce que todas las demas mediciónes
no puedan apreciarse correctamente.

\begin{figure}[H]
\hspace{1cm}\includegraphics[scale=0.5]{TwVsBw}\caption{Peso total en Kg Vs peso del cerebro en G{*}{*}}
\end{figure}


{*}{*} Se removieron los valores para los $4$ valores de $x$ mayores
a $2000$ ya que ocultaban la visualización de todos los demás valores

4)
\begin{lyxcode}
\begin{figure}[H]
\hspace{2cm}\includegraphics[scale=0.6]{logBrains}\caption{Logaritmo de ambas mediciónes y la línea que mejor los aproxima}
\end{figure}


\begin{figure}[H]
\hspace{2cm}\includegraphics[scale=0.6]{outLayerRemovidos}\caption{Logaritmo de ambas mediciónes y la línea que mejor los aproxima}
\end{figure}


\end{lyxcode}

\end{document}
