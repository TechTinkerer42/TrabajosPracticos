% Vista preliminar del código fuente

%% LyX 2.0.0 created this file.  For more info, see http://www.lyx.org/.
%% Do not edit unless you really know what you are doing.
\documentclass[english]{article}
\usepackage[T1]{fontenc}
\usepackage[latin9]{inputenc}

\makeatletter

%%%%%%%%%%%%%%%%%%%%%%%%%%%%%% LyX specific LaTeX commands.
%% Binom macro for standard LaTeX users
\newcommand{\binom}[2]{{#1 \choose #2}}

%% Because html converters don't know tabularnewline
\providecommand{\tabularnewline}{\\}

\makeatother

\usepackage{babel}
\begin{document}

\title{Aprendizaje Automatico - Trabajo Practico 3}


\author{Gonzalo Castiglione - 49138}

\maketitle

\paragraph*{Objetivo: Aplicar diversos métodos estadísticos para aprender a hacer
inferencia a partir de datos experiemtales.}


\section{Métodos de estadística paramétrica}
\begin{enumerate}
\item Soluciones

\begin{enumerate}
\item %
\begin{tabular}{|c|c|c|c|c|}
\hline 
 & Ancho & Largo & Ancho & Largo\tabularnewline
\hline 
\hline 
Media & 5.8433 & 3.0573 & 3.7580 & 1.1993\tabularnewline
\hline 
Varianza & 0.8281 & 0.4359 & 1.7653 & 0.7622\tabularnewline
\hline 
\end{tabular}
\item asd
\item asd
\end{enumerate}
\item Se tienen 80 componentes, de las cuales 12 son defectuosas.

\begin{enumerate}
\item La proporción de componentes no defetuosos de la muestra = $\frac{80-12}{80}=0.85$

\begin{enumerate}
\item asd
\item asd
\end{enumerate}
\item Proporción de sistemas que funcionan correctamente = $\frac{\binom{80-12}{2}}{\binom{80}{2}}=\frac{2278}{3160}=0.72$\end{enumerate}
\end{enumerate}

\end{document}
